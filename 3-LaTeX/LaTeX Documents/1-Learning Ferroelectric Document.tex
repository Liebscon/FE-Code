\documentclass{article}

\usepackage{color}
\usepackage{amsmath}

\title{Sample Ferroelectric Document}
\author{Tyler Liebsch}

\begin{document}
\linespread{1.3}
\maketitle

\section{Free Energy Analysis}

The phase transistions of a bulk crystal of BaTiO$_3$ can be analyzed by studying the free energy and its successive derivatives.  The equilibrium conditions can be determined by setting the first derivative equal to zero for each order parameter.

\section{Equations Studied}
\label{sec:Equations}

\subsection{Orthorhombic BaTiO$_3$}

The Free Energy can be broken down into various subsections:

\begin{align}
 F_{Polar} & =a_1 \sum _i P_i^2+a_{11} \sum _i P_i^4 + a_{12} \sum _{i<j} P_i^2 P_j^2 \\
 F_{Elastic} &= \frac{1}{2} c_{11} \sum _{i} u_{ii} + c_{12} \sum _{i<j} u_{\text{ii}} u_{j j}++\frac{1}{2} c_{44} \sum _{i j} u_{\text{ij}} \\
 F_{Electrostriction}& = -q_{11} \sum _i P_i^2 u_{\text{ii}}-q_{12} \sum _{i\neq j\neq k} u_{\text{ii}} \left(P_j^2+P_k^2\right)-q_{44} \sum _{i<j} P_i u_{\text{ij}} P_j \\
 F_{Gradient} &=\frac{1}{2} g_{11} \sum _i P_{i,i}^2+g_{12} \sum _{i<j} P_{i,i} P_{j,j}+\frac{1}{2} g_{44} \sum _{i<j} \left(P_{i,j}+P_{j,i}\right){}^2
\end{align}

The solutions for the orthorhombic phase of barium titanate have the following equlibrium solutions for the strain order parameter.
\vspace{12pt}


\begin{equation}
2+2=4
\label{eq1}
\end{equation}
Lets reference our first equation denoted by \eqref{eq1}

\section{Conclusion}





\end{document}