\documentclass[12pt]{article}

\usepackage{color}
\usepackage{amsmath}
% \usepackage{wasysym}
\usepackage{graphicx}

%New Command
\newcommand{\td}{\text{d}}

\title{Starting LaTeX}
\author{Tyler Liebsch}

\begin{document}
\linespread{1.4}
\oddsidemargin=15pt
\maketitle

\section{Learning Math}

\begin{itemize}
\item \$\$
  $$
     5+3=8
  $$

\item equations
  \begin{equation}
    a+b=c
    \label{eq1}
  \end{equation}
  \begin{equation}
  2+2=4
  \end{equation}
Lets reference our first algebraic equation denoted by \eqref{eq1}

\item align*
    \begin{align*}
    a+b & =c \\
    g*f-24+5 & = 37
    \end{align*}
\item{align}
   \begin{align}
   \nonumber
   a+b & = c \\
   \nonumber
   b & = c-a \\
  \label{eq2}
   \frac{1}{b}*{b} & = \frac{c-a}{b}
   \end{align}
It follows from \eqref{eq2} that \eqref{eq1} is true.
\end{itemize}

\newpage

\section{The Heat Exchange Equation.}
This derivation is by Bernd Schr\*oder of Louisiana State.

\begin{align}
% Ampersands alternate between alignment characters and spacers!
    -\iint_S\nabla \cdot \nabla u \cdot \text{d}\vec{S}
    & = -k \frac{\partial}{\partial t} \iiint_B u \, \text{d}v \\
    \iint_S \nabla u \cdot \td \vec{S}
    & = \iiint_B k \frac{\partial u}{\partial t} \, \td V \\
    \iiint_B \nabla \cdot \nabla u \, \td V & = \iiint_B k \frac{\partial u}{\partial t} \td V \\
    \lim_{a \rightarrow 0}\frac{1}{ \frac{4}{3} \pi a^3} \iiint_B \nabla \cdot \nabla u \, \td V  
    &= \lim_{a \rightarrow 0}\frac{1}{ \frac{4}{3} \pi a^3} \iiint_B k \frac{\partial u}{\partial t} \td V \\
    \nabla \cdot \nabla u( \vec{r},t) &= k \frac{\partial u}{\partial t}(\vec{r}, t)
\end{align}
% Usable Commands are pmatrix, bmatrix, matrix
% pmatrix has brackets around the matrix, the normal matrix command has no brackets.
\section{Matrices}
$$ \begin{pmatrix}
a & b & c \\
d & e & f \\
g & h & i 
\end{pmatrix} $$

\section{Table \& Figure Example}

In Table \ref{tab:classlist}, you will find the list of class participatns.

\begin{table}[htb]
% The Label command for the table has to come after the caption.
\begin{center}
\begin{tabular}{|l|l|l|}
\hline
Name & Department & Email \\ 
\hline \hline
Luke Corwin & Physics & luke.corwin@sdsmt.edu\\
Michelle While & Physics & $ \int e^x \, \text{d}x =x$ \\
Deb Bienert & MCS & $1.53 \pm 0.3 $\\
Tyler Liebsch & Physics & tyler.liebsch@mines.sdsmt.edu\\
\hline
\end{tabular}
\caption{List of people in this class}
\label{tab:classlist}
\end{center}
\end{table}

\newpage
% Tilda will create a space without ever breaking a line.
In Figure ~\ref{fig:lattice}, the difference in the crystalline structure can be seen.

\begin{figure}
\caption{Crystal Lattice Structures}
\includegraphics[width=0.5\textwidth]{C:/Users/Tyler/Desktop/Crystal-Lattices}
\label{fig:lattice}
\reflectbox{\includegraphics[width=0.5\textwidth]{C:/Users/Tyler/Desktop/Crystal-Lattices}}
\end{figure}


\end{document}