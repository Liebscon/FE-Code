\documentclass{article}

\usepackage{color}
\usepackage[fleqn]{amsmath}
%\usepackage{mathtools}
\usepackage{graphicx}
\usepackage{tipa}
\usepackage{gensymb}
\usepackage{ragged2e}
\usepackage{grffile}
\usepackage[margin=1.25in]{geometry}
%\usepackage{parskip}
%\setlength{\parindent}{15pt}

\begin{document}
\linespread{1.0}
\justify

%===========================================================================================
\section{Free Energy Analysis}

%\begin{split}\Phi &=a_1 \left(P_1^2+P_2^2+P_3^2\right)+a_{11} \left(P_1^4+P_2^4+P_3^4\right)+a_{12} \left(P_1^2 P_2^2+P_3^2 P_2^2+P_1^2 P_3^2\right)+ \\ 
%&a_{111} \left(P_1^6+P_2^6+P_3^6\right)+a_{112} \left( P_1^4\left(P_2^2+P_3^2\right)+P_2^4\left(P_1^2+P_3^2\right)+P_3^4 %\left(P_2^2+P_1^2\right)\right)+ \\ 
%& a_{123} P_1^2 P_2^2 P_3^2+ \\ 
%&\frac{1}{2} c_{11} \left(u_1^2+u_2^2+u_3^2\right)+ c_{12} \left(u_1 u_2+u_3 u_2+u_1 u_3\right)+\frac{1}{2} c_{44} \left(u_4^2+u_5^2+u_6^2\right)- \\ 
%& q_{11} \left(P_1^2 u_1+P_2^2 u_2+P_3^2 u_3\right)-q_{12} \left(P_1^2 \left(u_2+u_3\right)+P_3^2 \left(u_1+u_2\right)+P_2^2 \left(u_1+u_3\right)\right)- \\ & %q_{44} \left(P_2 P_3 u_4+P_1 P_3 u_5+P_1 P_2 u_6\right) \end{split} \\

The Free Energy is given below using an eighth order expansion of the polarization order parameter. \\
\begin{align}
\Phi &= \Phi_{LG}+\Phi_c+\Phi_q+\Phi_{grad} \\
\Phi_{LG} &= a_1 \left(P_1^2+P_2^2+P_3^2\right)+a_{11} \left(P_1^4+P_2^4+P_3^4\right)+a_{12} \left(P_1^2 P_2^2+P_3^2 P_2^2+P_1^2 P_3^2\right)+ \\ 
&a_{111} \left(P_1^6+P_2^6+P_3^6\right)+a_{112} \left( P_1^4\left(P_2^2+P_3^2\right)+P_2^4\left(P_1^2+P_3^2\right)+P_3^4 \left(P_2^2+P_1^2\right)\right)+ \nonumber \\  \nonumber
& a_{123} P_1^2 P_2^2 P_3^2 +a_{1111} \left(P_1^8+P_2^8+P_3^8\right)+a_{1122} \left(P_1^4 P_2^4+P_3^4 P_2^4+P_1^4 P_3^4\right)+\\
&a_{1112} \left(\left(P_2^2+P_3^2\right) P_1^6+\left(P_2^6+P_3^6\right) P_1^2+P_2^2 P_3^6+P_2^6 P_3^2\right)+a_{1123} P_1^2 P_2^2 P_3^2 \left(P_1^2+P_2^2+P_3^2\right)
\nonumber \\ \Phi_{c} &=\frac{1}{2} c_{11} \left(u_1^2+u_2^2+u_3^2\right)+ c_{12} \left(u_1 u_2+u_3 u_2+u_1 u_3\right)+\frac{1}{2} c_{44} \left(u_4^2+u_5^2+u_6^2\right) \\
\Phi_{q} &= -q_{11} \left(P_1^2 u_1+P_2^2 u_2+P_3^2 u_3\right)-q_{12} \left(P_1^2 \left(u_2+u_3\right)+P_3^2 \left(u_1+u_2\right)+P_2^2 \left(u_1+u_3\right)\right)- \\
& q_{44} \left(P_2 P_3 u_4+P_1 P_3 u_5+P_1 P_2 u_6\right) \nonumber
\end{align}
The sixth order expansions are shown below without Voigt notation:
\begin{align*}
\Phi_{P} &= a_1 \left(P_1^2+P_2^2+P_3^2\right)+a_{11} \left(P_1^4+P_2^4+P_3^4\right)+a_{12} \left(P_1^2 P_2^2+P_3^2 P_2^2+P_1^2 P_3^2\right) \\ 
& \qquad + a_{111} \left(P_1^6+P_2^6+P_3^6\right)+a_{112} \left( P_1^4\left(P_2^2+P_3^2\right)+P_2^4\left(P_1^2+P_3^2\right)+P_3^4 \left(P_2^2+P_1^2\right)\right) \nonumber \\ \nonumber & \qquad + a_{123} P_1^2 P_2^2 P_3^2 \\
\Phi_{E}&=\frac{1}{2} c_{11} \left(u_{11}^2+u_{22}^2+u_{33}^2\right)+ c_{12} \left(u_{11} u_{22}+u_{33} u_{22}+u_{11} u_{33}\right)+\frac{1}{2} c_{44} \left(u_{23}^2+u_{13}^2+u_{12}^2\right) \\
\Phi_{Q} &= -q_{11} \left(P_1^2 u_{11}+P_2^2 u_{22}+P_3^2 u_{33}\right)-q_{12} \left(P_1^2 \left(u_{22}+u_{33}\right)+P_3^2 \left(u_{11}+u_{22}\right)+P_2^2 \left(u_{11}+u_{33}\right)\right) \\
 & \qquad -q_{44} \left(P_2 P_3 u_{23}+P_1 P_3 u_{13}+P_1 P_2 u_{12}\right) \nonumber \\
 \Phi_{G} &= \frac{\delta_{11}}{2}(P_{1,1}^2+P_{1,1}^2+P_{1,1}^2)+\delta_{12}(P_{1,1} P_{2,2}+P_{1,1} P_{3,3}+ P_{2,2} P_{3,3}) \\
 & \qquad +\frac{\delta_{44}}{2}[(P_{1,2}+P_{2,1})^2+(P_{1,3}+P_{3,1})^2+(P_{2,3}+P_{3,2})^2]
\end{align*}

Our phenomenological model studies a single domain barium titanate crystal in the absence of an electric field and any applied stress.There exist several solutions to the minimzation of the free energy for ferroelectric perovskites.  Here we discuss the tetragonal to monoclinic and subsequent orthorhombic phase transition sequence.  The free energy must be expanded up to the eighth order in order for the monoclinic solution to be a observed.

\label{sec:Formulation}
\begin{align}
\frac{\partial \Phi}{\partial u_{ij}}=0 \to u_{ij}[P]\to \frac{\partial \Phi}{\partial P_i}=0 \to P_i[T]
\end{align}

\section{Strain Relations}
\label{sec:Strain}
The first derivative of the free energy gives the strain equilbrium conditions.
\begin{align}
u_{11} &= Q_{11} P_1^2 + Q_{12}(P_2^2 + P_3^2) \\
u_{22} &= Q_{11} P_2^2 + Q_{12}(P_1^2 + P_3^2) \\
u_{33} &= Q_{11} P_3^2 + Q_{12}(P_1^2 + P_2^2) \\
u_{23} &= Q_{44}P_2P_3 \\
u_{13} &=  Q_{44}P_1P_3 \\
u_{12} &=  Q_{44}P_1P_2
\end{align}

The quadratic coefficients are then renormalized as:
\begin{align}
a'_{11} &=  a_{11}-( \frac{q_{11} Q_{11}}{2}+q_{12}Q_{12}) \\
a'_{12} &= a_{12}-(q_{12} Q_{11}+q_{11} Q_{12}+q_{12} Q_{12}+\frac{q_{44} Q_{44}}{2}) \\
\end{align}
It is important to note that the coefficients given in papers are derived from experiments are analyzed using first principles to fit experimental observations.  As such the parameters are already normalized to include the contribution from strain.


%===========================================================================================
\newpage
\section{Polarization}

The temperature-dependence of the polarization function can be determined by calculating the solutions to the derivative of the free energy as described by Section \ref{sec:Formulation}.  The analytical form for the sixth-order solutions corresponding to the tetragonal and orthorhombic are shown below.

\begin{align}
\frac{\partial \Phi}{\partial P_i}=0 \to P_i[T]
\end{align}

\subsection{Tetragonal Polarization}
\begin{equation}
P_3^2[T]=\frac{\sqrt{a_{11}^2-3 a_1 a_{111}}-a_{11}}{3 a_{111}}
\end{equation}

\subsection{Orthorhombic Polarization}
\begin{equation}
P^2[T]=\frac{-2 a_{11}-a_{12}+\sqrt{4 a_{11}^2+4 a_{12} a_{11}+a_{12}^2-12 a_1 \left(a_{111}+a_{112}\right)}}{3(a_{111}+a_{112})}
\end{equation} \\

The temperature dependence of the polarization for the T, O, and M$_C$ are shown below. \\
\includegraphics[width=1.0\linewidth]{C:/Users/Tyler/Desktop/LaTeXFigures/Polarization Magnitude T,O,Mc Phases.pdf} \\
\includegraphics[width=1.0\linewidth]{C:/Users/Tyler/Desktop/LaTeXFigures/Polarization Components T,O,Mc Phases.pdf}


%===========================================================================================
\newpage
\section{Renormalized Free Energy}

The elastic field resulting in strains on the lattice structure leads to the renormlization of the thermodynamic parameters. \\
\center $ \Phi_{Relaxed} = \Phi_0 + \delta \Phi $ \\
\flushleft
Tetragonal Phase
\begin{flalign*}
\Phi_T &= a_1 P_1^2 + a'_{11} P_1^4 + a_{111} P_1^6 &\\
\Phi_T &= a_1 P_1^2 + (a_{11}-\delta a_{11}) P_1^4 + a_{111} P_1^6 &\\
\delta \Phi_T &= \delta a_{11} P_1^4 &
\end{flalign*}
Orthorhombic Phase \\
Let $\frac{1}{2} P^2=P_1^2=P_3^2 $
\begin{flalign*}
\Phi_O &=a_1 P^2+ \frac{1}{4} \left(2 a'_{11}+a'_{12}\right) P^4+\frac{1}{4} \left(a_{111}+a_{112}\right) P^6 &\\
\Phi_O &=a_1 P^2+ \frac{1}{4} \left(2a_{11}+a_{12}-2\delta a_{11}-\delta a_{12}\right) P^4+\frac{1}{4} \left(a_{111}+a_{112}\right) P^6 &\\
\delta \Phi_O &= \frac{1}{4}(2\delta a_{11} + \delta a_{12})P^4 &
\end{flalign*}

Monoclinic Phase 
\begin{flalign*}
\Phi_{Mc} &=a_1 \left(P_1^2+P_3^2\right)+a'_{11} \left(P_1^4+P_3^4\right)+a'_{12} P_1^2 P_3^2+\\ 
                 & \quad + a_{111} \left(P_1^6+P_3^6\right)+a_{112} P_1^2 \left(P_1^2+P_3^2\right) P_3^2 &\\
\Phi_{Mc} &= a_1 \left(P_1^2+P_3^2\right)+(a_{11}-\delta a_{11}) \left(P_1^4+P_3^4\right)+(a_{12}-\delta a_{12}) P_1^2 P_3^2 +\\
                 & \quad +a_{111} \left(P_1^6+P_3^6\right)+a_{112} P_1^2 \left(P_1^2+P_3^2\right) P_3^2 &\\
\delta \Phi_{Mc} &= \delta a_{11} \left(P_1^4+P_3^4\right)+\delta a_{12}P_1^2 P_3^2 &
\end{flalign*}

Using the coefficients determined by [26].
\begin{align}
\delta a_{11}=1.56507*10^9 \qquad
\delta a_{12}=-1.3248*10^9
\end{align}
\begin{align*}
\delta \Phi_T &\geq 0 \qquad \delta \Phi_O \geq 0 \qquad \delta \Phi_{Mc} \geq 0
\end{align*}

\newpage
\subsection{Phase Transition Temperatures}
The phase transition temperature can be determined by comparing the Gibbs free energy with the lowest determining the most thermodynamically favorable state.  The phase transtition from the tetragonal to the cubic phase and the tetragonal to orthorhombic phase are shown below.

\begin{align}
\Phi_{Cubic} &= \Phi_{Tetra.} \to T_C = 398 \degree K (125 \degree C) \\
\Phi_{Tetra.} &= \Phi_{Ortho.} \to T_{T-O} = 281 \degree K (8\degree C)
\end{align}

The free energy for the monoclinic phase is greater than both the tetragonal and orthorhombic free energy; however, an applied electric field could provide enough of a decrease in the free energy if properly aligned with the principal monoclinic polarization direction.  Since the monoclinic phase  has a higher free energy it can only appear as a metastable phase or a distortion of the transitioning unit cell.

\includegraphics[width=1.0\linewidth]{C:/Users/Tyler/Desktop/LaTeXFigures/FreeEnergy(T) T,O,Mc Phases.pdf}
\includegraphics[width=1.0\linewidth]{C:/Users/Tyler/Desktop/LaTeXFigures/FreeEnergy(T) T,O,Mc Phases(ZoomedIn).pdf}

\newpage
\subsection{Stability Range}

The phase transition from the tetragonal structure to the orthrohombic one is a first order phase transition; therfore, the existence of metastable states near the point of transition must be studied.  By analyzing the second derivative of the free energy for each phase the range of stabliity can be determined.

\begin{align}
\frac{\partial^2 \Phi}{\partial P_i \partial P_j} \geq 0
\end{align}

The positive temperature range of this determinant gives the range of stability and is plotted below.  The monoclinic phase is never positive however approaches zero at $T \simeq 309 \degree K$.  The overlapping regions of stability for the tetragonal and orthorhombic phases describes a first order phase transition.

\includegraphics[width=1.0\linewidth]{C:/Users/Tyler/Desktop/LaTeXFigures/Stability Range T,O,Mc Phases.pdf}

%===========================================================================================
\newpage
\justify

\section{Entropy of Phase Transitions}
\begin{align*}
\Delta S &= \frac{2 \pi}{C}\Delta P^2
\end{align*}

The entropy for the well-known phase transitions of barium titanate are compared with the results of Jona and Shirane.
\vspace{12 pt}

\begin{tabular}{| l | c | c| c|}
     \hline
     Phase Transition &   $\Delta S$                    &   $\Delta$ S$^1$             &   $\Delta$ S (Range)$^{2,3,4}$ \\
                                &  (cal/mole $\degree K$)  & (cal/mole $\degree K$)   & (cal/mole $\degree K$)  \\ \hline
     C-T                      & 0.125                              & 0.125                            & 0.12 $ \sim $ 0.125 \\ \hline
     T-O                     & 0.074                             & 0.076                             & 0.054 $ \sim $ 0.091 \\ \hline
\end{tabular} \\

\vspace{12 pt}

The entropy for the monoclinic phase transition sequence is shown below.  Since the phase transition temperature could not be determined using Landau theory, the experimental phase transition temperature  $ (T \simeq 283 \degree K )^5 $ was used to analyze the change in entropy.
\vspace{12 pt}

 \begin{tabular}{| l |c |}
        \hline
        Phase Transition & $\Delta S$ \\
        &  (cal/mole $\degree K$) \\ \hline
         T-M$_c$ &0.032\\ \hline
         M$_c$-O & 0.040\\ \hline
\end{tabular}

\begin{align}
S=- \Big[ \frac{\partial \Phi}{\partial T}\Big] _P
\end{align}
The thermodynamic parameters used here were determined by Li \textit{et al.}.  The quadratic coefficient in the free energy is the only temperature dependent term thus the entropy for all three phases takes the form:
\begin{align}
\alpha_1 &=a_0(T-T_C) \\
S &= -4.124E5(P_1^2+P_2^2+P_3^2)
\end{align}

\includegraphics[width=1.0\linewidth]{C:/Users/Tyler/Desktop/LaTeXFigures/Entropy(T) T,O,Mc Phases.pdf}

\footnotetext[1]{Jona and Shirane \textit{Ferroelectric Crystals}}
\footnotetext[2]{Shirane and Takeda}
\footnotetext[3]{Volger}
\footnotetext[4]{Blattner \textit{et al.}}
\footnotetext[5]{Eisenschmidt \textit{et al.}}

%===========================================================================================
\newpage
\section{Discussion}
\justify

\indent 	The equilbrium equations for the strain and the spontaneous polarization were analzed in the absence of an electric field or applied stress.  The resulting solutions gave the temperature depenence of these order parameters.  The temperature dependence of the polarization for the tetragonal phase and the othorhombic phase were in agreement with previous phenomenological derivations and experimental results.  The temperature interval of stability as well as the change in entropy for each phase transition was also determined. \\
\indent 	The behavior of the magnitude of the polarization vector for the O and M$_c$ phases describes a 2$^{nd}$ order phase transition since the order parameter is continuous about T$ \simeq 309 \degree K $.  This point is above the phase transition temperature from the tetragonal phase to the orthorhomic phase; thus when the O phase becomes metastable, within this temperature interval T $ < 309 \degree K $, the Monoclinic phase seen experimentally represents a distortion of the O lattice and therefore the polarization vector. The apperance of the Monoclinic lattice structure increases the overall entropy of the crystal; however, since the O phase is only metastable over the temperature interval 281 $ < T < 309 \degree K$, the monoclinic phase, acting as a distortion of the O phase, must also remain metastable.  Furthermore, we have shown that the free energy of the monoclinic phase remains greater than the free energy of the T and O phases. \\


\section{Scratch Work}
\begin{align}
\Phi=\Phi_0 + \frac{1}{2}\alpha P_i^2 + \frac{1}{2}\beta P_i^4 +\frac{1}{6} \gamma P_i^6 \\
\\
P_i &=0 \\
P_i &= \pm P_s = \pm \sqrt{\alpha_0(T_0 - T)/ \beta}
\end{align}

\newpage
\section{Domain Structures}
\justify

\indent           The appearance of polarization about the phase transition gives rise to two orientations of the polarization vector.  Domains existing within the crystalline structure can be considered from a phenomenological standpoint from the change in sign of the spontaneous polarization vector.  Domains can be observed experimentally as regions of uniformly orientated polarization vector.  Domain walls are formed at the interface between these regions.
The Gibbs free energy including the spatial gradients of the order parameter are given below.

The Strain and Electrostriction contributions to the free energy are not included since these expressions can be accounted for by renormalizing the coefficients of the order parameter as seen in previous sections.

\begin{align*}
\Phi_{P} &= a_1 \left(P_1^2+P_2^2+P_3^2\right)+a_{11} \left(P_1^4+P_2^4+P_3^4\right)+a_{12} \left(P_1^2 P_2^2+P_3^2 P_2^2+P_1^2 P_3^2\right) \\ 
& \qquad + a_{111} \left(P_1^6+P_2^6+P_3^6\right)+a_{112} \left( P_1^4\left(P_2^2+P_3^2\right)+P_2^4\left(P_1^2+P_3^2\right)+P_3^4 \left(P_2^2+P_1^2\right)\right) \nonumber \\ \nonumber & \qquad + a_{123} P_1^2 P_2^2 P_3^2 \\
 \Phi_{G} &= \frac{\delta_{11}}{2}(P_{1,1}^2+P_{1,1}^2+P_{1,1}^2)+\delta_{12}(P_{1,1} P_{2,2}+P_{1,1} P_{3,3}+ P_{2,2} P_{3,3}) \\
 & \qquad +\frac{\delta_{44}}{2}[(P_{1,2}+P_{2,1})^2+(P_{1,3}+P_{3,1})^2+(P_{2,3}+P_{3,2})^2]
\end{align*}

\indent The tetragonal phase has the simplest structure of polarization.  The elastic and electrostatic equilbrium conditions for the strain field leads to the domain walls forming either a 90$\degree $ or 180$\degree$ boundary.   This angle compares the polarization vector on either side of the domain wall.

\subsection{Background}
\subsection{Elastic Contribution}
\subsection{Simple Tetragonal Calculation (180$\degree$ D.W.)}
\subsection{90$\degree$ Domain Wall}
\subsection{Martensitic Phase Transformations}
\subsection{Twinning}
\section{Additional Approaches}
\subsection{Structural Phase Transitions}
\subsection{Statistical Theory-Hamiltonian}

Continued in Latex File  Number 6




\end{document}